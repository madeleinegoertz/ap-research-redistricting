\section{Conclusion}
\label{sec:conc}

The objective of this research was fill the gap in the scholarship with respect to comparative, empirical case studies of the performance of automated redistricting algorithms. I therefore set out to answer the question, \emph{how do the hypothetical district maps for the Virginia Congressional delegation for the 2020s generated by different automated redistricting algorithms compare based on partisan fairness and compactness measures?}

In my experimental case study, I simulated the 2021 redistricting of the congressional districts in Virginia using the CRSG and SMC algorithms, and evaluated the resulting maps and the existing map using various measures of compactness\footnote{The Polsby-Popper score and the edge-cut compactness score} and of partisan fairness\footnote{The seats-votes curve, partisan bias, the efficiency gap, declination, and the mean-median difference}. 

The key conclusions from my research are that SMC reliably generated the fairest districts, that CRSG generated fair districts less reliably, and that the existing district map significantly favored Republicans. This suggests that SMC is the favored choice of algorithm for such a redistricting task. Both algorithms and the existing map were reasonably compact.

Regarding compactness measures, I confirmed the findings of \textcite{mccartan2020} that the Polsby-Popper measure fails to assess compactness fairly when physical geography necessitates a certain degree of incompactness. I find the edge-cut compactness measure to be robust against variations in physical geography, which confirms the findings of \textcite{dube2016}.

Regarding measures of partisan fairness, the seats-votes curve provided the clearest picture of irregularities in the seats-votes relationship, which agrees with the conclusions of \textcite{katz2020}. I also find the mean-median difference to align with the partisan bias measure, as theoretically verified by \textcite{katz2020}. My results also confirmed the findings of \textcite{veomett2018} that the efficiency gap can measure partisan symmetry in competitive elections. Lastly, I also find declination to measure the skewness in the district vote distribution, not partisan fairness, confirming the theory of \textcite{katz2020}. 

Taken in concert, my findings present a significant evolution in the capabilities of automated redistricting algorithms, as well as initial evidence that the SMC algorithm could successfully be used by a redistricting commission tasked with redrawing district boundaries. Tangentially, I also provided limited empirical evidence supporting the findings of various scholars that certain measures either do or do not assess deviation from partisan symmetry.

As the SMC algorithm continues to evolve and improve, we could be approaching a reality where a number of fair redistricting maps are generated by algorithms, and then the final map is selected by members of a committee that represents various stakeholders.

However, further research is needed to continue evaluating redistricting algorithms as they continue to improve. My research only focused on a single redistricting cycle of a single delegation of a single state. Furthermore, access to greater computing power would enable the generation of larger sets of maps and for maps with more districts. Additionally, empirical evaluations of these algorithms will need to include further real-world stipulations, such as obeying county boundaries, preserving majority-minority districts, and generating districts that vary minimally from the existing districts. 

Such research will enable the transition of automated redistricting algorithms from the scholarly literature into practice in independent commissions, which is one necessary step towards the goal of increasing fairness in our district-based democracy. 