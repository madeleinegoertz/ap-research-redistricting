\section{Results}
\label{sec:results}

This section provides an overview of the results from my method. 

\subsection{Compactness}
\begin{figure}
    \centering
    \caption{Distribution of Compactness Measures}
    \includegraphics[width=\textwidth]{img/compact.density.png}
    \label{fig:compact.density}
    \raggedright
    \figurenote{The first column of plots shows the \hyperref[sec:polsbypopper]{Polsby-Popper} score distribution; the second column shows the \hyperref[sec:edgecut]{Edge-Cut compactness} distribution. The first row corresponds to the plans generated by SMC; the second row corresponds to CRSG. The red lines indicate the measure value for the existing districts; the green lines indicate the mean value of the distribution.}
\end{figure}

Figure \ref{fig:compact.density} visualizes the distribution of two different compactness scores amongst plans generated by both SMC and CRSG. Subfigure
\begin{seriate} 
    \item shows the distribution of the \hyperref[sec:polsbypopper]{Polsby-Popper scores} of the 100 SMC plans, subfigure
    \item shows the distribution of the \hyperref[sec:edgecut]{Edge-Cut compactness measure} of the same 100 SMC plans, and subfigures
    \item and 
    \item show the corresponding measure distributions for the plans generated by CRSG. 
\end{seriate}
The vertical red lines indicate the value of the measure for the existing district map, and the vertical green lines indicate the mean value of the distribution. 

\subsection{Partisan Fairness}

\subsubsection{Seats-Votes Curves}
\begin{figure}
    \centering
    \caption{Seats-Votes Curves for SMC, CRSG, and existing plan.}
    \begin{subfigure}[b]{0.475\textwidth}
        \centering
        \includegraphics[width=\textwidth]{img/sv.smc.png}
        \caption{SMC Seats-Votes Curve}
        \label{fig:sv.smc}
    \end{subfigure}
    \hfill
    \begin{subfigure}[b]{0.475\textwidth}
        \centering
        \includegraphics[width=\textwidth]{img/sv.crsg.png}
        \caption{CRSG Seats-Votes Curve}
        \label{fig:sv.crsg}
    \end{subfigure}
    \vskip\baselineskip
    \begin{subfigure}[b]{0.475\textwidth}
        \centering
        \includegraphics[width=\textwidth]{img/sv.control.png}
        \caption{Real Seats-Votes Curve}
        \label{fig:sv.control}
    \end{subfigure}
    \label{fig:sv}
    \raggedright
    \figurenote{Each plot shows the relationship between average proportion of Democratic vote share by district and the proportion of Democratic seats. Subfigure \ref{fig:sv.smc} illustrates this relationship for the 100 plans generated by SMC, subfigure \ref{fig:sv.crsg} for CRSG, and subfigure \ref{fig:sv.control} is for the existing plan.}
\end{figure}

Figure \ref{fig:sv} shows the seats-votes curves \parencite{katz2020} for the 2018 General Election under the redistricting plans generated by both algorithms and the existing map. For each plot, the x-axis plots the average of the proportion of votes won by Democrats in each district. The y-axis plots the proportion of seats won by Democrats in the delegation. Both subfigures \ref{fig:sv.smc} and \ref{fig:sv.crsg} have once curve for each redistricting plan (each plot has 100 curves). The seats-votes curve for the real 2018 districts is provided for reference in subfigure \ref{fig:sv.control}.

\subsubsection{Single-Valued Partisan Fairness Measures}
\begin{landscape}
    \begin{figure}
        \centering
        \caption{Distributions of fairness measures}
        \includegraphics{img/fair.density.png}
        \label{fig:fair.density}
        \raggedright
        \figurenote{The columns of plots illustrate the distributions of \hyperref[sec:bias]{partisan bias}, the \hyperref[sec:effgap]{efficiency gap}, \hyperref[sec:declination]{declination}, and the \hyperref[sec:meanmed]{mean-median difference}, respectively. The first row of plots corresponds to SMC plans, the second to CRSG plans. The red lines indicate the corresponding value of the measure in the existing plan; the green lines indicate the mean value of the distribution.}
    \end{figure}
\end{landscape}

Figure \ref{fig:fair.density} illustrates the distributions of various fairness measures within the plans generated by SMC and CRSG. Columns of plots correspond to different measures, and rows of plots correspond to different algorithms. The red and green vertical lines indicate the control value and mean distribution value, respectively. 

\subsection{2018 General Election Simulation}

\subsubsection{Electoral Maps}
\begin{figure}
    \centering
    \caption{Simulated 2018 Virginia Congressional Election under SMC and CRSG}
    \includegraphics[width=0.7\textwidth]{img/election.map.png}
    \label{fig:election.map}
    \raggedright
    \figurenote{Each map uses the district plan with the least-magnitude partisan bias for SMC, CRSG, and the existing plan, respectively. Navy corresponds to a greater Democratic vote proportion, and red corresponds to a greater Republican vote proportion.}
\end{figure}

Figure \ref{fig:election.map} visualizes the outcome of a simulated 2018 General Election under various redistricting plans. Subfigure 
\begin{seriate} 
    \item  was created by aggregating the precinct-level election results from 2018 to the district level using the redistricting plan generated by SMC with the least magnitude partisan bias. The average precinct-level proportion of votes won by Democrats in the district is displayed as a percentage. Values closer to 1 indicate a higher proportion of Democratic votes and are colored blue. Values closer to 0 indicate a higher proportion of Republican votes and are colored red. Purple districts are most competitive. Subfigure 
    \item illustrates this same simulation, only using the corresponding "fairest" redistricting plan generated by SMC. The real results from the 2018 General Election are visualized in subfigure 
    \item for reference. 
\end{seriate}

\subsubsection{Seats distribution}

\begin{figure}
    \centering
    \caption{Distributions of Democratic seats under SMC and CRSG}
    \includegraphics[width=\textwidth]{img/dseats.hist.png}
    \label{fig:dseats.hist}
    \raggedright
    \figurenote{a) shows the distribution of the number of seats allocated to Democrats by the 100 SMC plans; b) shows the corresponding distribution for CRSG plans. The red and green lines indicate the control and mean distribution value of the number of Democratic seats, respectively.}
\end{figure}

Finally, Figure \ref{fig:dseats.hist} illustrate the distribution of seats allocated to Democrats by each algorithm, as well as the true number of seats from 2018 (indicated by the red line). 