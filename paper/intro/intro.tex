% Manually putting in the title here and using donotrepeattitle in paper.tex so that I don't have a section header for "Introduction", but there is still a section in the VScode navigation for all the intro subsections to live within. 
\makeatletter
\section{\@title}
\makeatother

Legislative redistricting is the process of drawing district boundaries for elections. In the United States, the redrawing of congressional district boundaries occurs every 10 years after the census. This process of redistricting is generally performed by state legislatures, hybrid commissions, or independent commissions \parencite{princetongerrymanderingproject}. 

When redistricting congressional districts, the two most fundamental requirements are that the districts must be "reasonably compact" and of equal population, meaning that the population of each district must be within $$\pm$$ 1\% of the average district population \parencite{1964}. 

Gerrymandering is drawing the district boundaries in such a way as to unfairly benefit a particular interest group. There are two primary methods of gerrymandering: "packing" and "cracking." The US Supreme Court defines these as creating districts where the opposing party holds far above the majority or is spread out amongst so many districts as to become ineffectual, respectively \parencite{1986}.

Automated redistricting algorithms are one proposed solution to prevent gerrymandering. As early as 1961, such "automatic and impersonal procedure[s]" were thought to "produce results not markedly inferior to those which would be arrived at by a genuinely disinterested commission" \parencite[110]{vickrey1961}.

Broadly speaking, the goal of such algorithms is to create a map (or a selection of maps) that is "fair." Some algorithms obey only equal population and compactness constraints \parencite[e.g.][]{hu1995, altman2009, chen2013}, while more-modern algorithms have evolved to address real-world considerations, such as majority-minority districts, incumbency, and competitiveness \parencite[e.g][]{lara-caballero2019, fifield2020, mccartan2020}. 

In parallel to the development of these algorithms, measures of compactness and of partisan fairness were also developed. These compactness measures include geometric measures \parencite[e.g.][]{polsby1991, schwartzberg1966,harris1964, maceachren1985, reock1961, boyce1964} and measures from graph theory \parencite[e.g.][]{dube2016}. Proposed measures of partisan fairness generally try to characterize the relationship between the number of votes and the number of seats won by a political party. Some measures seek to simulate the power balance in the legislature at all possible party votes shares \parencite[e.g.][]{tufte1973}, while others aim to compute a single metric \parencite[e.g.][]{stephanopoulos2014,katz2020,warrington2018,mcdonald2015,wang2016}. Neither the scholarly literature nor the courts agree on a single correct compactness or partisan fairness measure. 

While the scholarship is focused on the development of new automated redistricting algorithms \parencite{fifield2020}, on the evaluation of the mathematical rigor of proposed metrics \parencite[see][]{katz2020}, and on the detection of gerrymandering in existing redistricting plans \parencite[e.g.][]{herschlag2017, duchin2018a}, research into the empirical performance of these algorithms in a real-world redistricting scenario is lacking \parencite{fifield2020a}.

Identifying this gap led me to my research question, \emph{How do the hypothetical district maps for the Virginia Congressional delegation for the 2020s generated by different automated redistricting algorithms compare based on partisan fairness and compactness measures?}

I chose Virginia as my case state because 
\begin{seriate}
    \item it has had to redistrict mid-decade due to gerrymandering \parencite[see][]{2016a},
    \item it is a competitive state from an electoral perspective, 
    \item it recently approved a constitutional amendment that institutes a redistricting process driven by a hybrid commission \parencite{ballotpediastaff2020}, and
    \item it has a complex geography.
\end{seriate}

With the goal of comparing the performance of different automated redistricting algorithms, I chose two algorithms of different levels of complexity that could both meet the real-world requirements of equal population and compactness. The Compact Random Seed Growth algorithm (CRSG) \parencite{chen2013} and the Sequential Monte Carlo algorithm (SMC) \parencite{mccartan2020} were chosen as they both satisfy the aforementioned requirement, are implemented by an open-source software package \parencite{fifield2020d}, and represent the evolution in the scholarship on such algorithms. 

Since there is no single measure of partisan fairness nor compactness agreed upon by the scholarship, I selected representative samples of the different perspectives on each measure. For compactness, one geometry-based measure (the Polsby-Popper score \parencite{polsby1991}) and one graph-theory-based measure (the edge-cut compactness score \parencite{dube2016}) were chosen. For partisan fairness, the simulated election measure (the seats-votes curve \parencite{tufte1973}) and four single-valued fairness measures (partisan bias \parencite{katz2020}, the efficiency gap \parencite{stephanopoulos2014}, declination \parencite{warrington2018}, and the mean-median difference \parencite{mcdonald2015}) were chosen. Together, these five partisan fairness measures represent differing and disputed views on how "fairness" should be quantified \parencite{katz2020}. 

I have chosen to conduct a case-study using the experimental research method, as it allows me to infer the causality of the different redistricting algorithms on the redistricting plan outcomes. 

This paper begins with a \hyperref[sec:litreview]{literature review} of the scholarship on automated redistricting algorithms, compactness measures, and partisan fairness measures. Further details of my chosen research method are explained in the \hyperref[sec:method]{Method} section. The results of my analysis are presented, interpreted, and reflected upon in the \hyperref[sec:results]{Results}, \hyperref[sec:disc]{Discussion}, and \hyperref[sec:conc]{Conclusion} sections, respectively. 